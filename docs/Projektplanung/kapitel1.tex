%%%%%%%%%%%%%%%%%%%%%%%%%%%%%%%%%%%%%%%%%%%%%%%%%%%%%%%%%%%%%%%%%%%%%%%%%%%%%%
%% Descr:       Vorlage für Berichte der DHBW-Karlsruhe, Ein Kapitel
%% Author:      Prof. Dr. Jürgen Vollmer, vollmer@dhbw-karlsruhe.de
%% $Id: kapitel1.tex,v 1.24 2020/03/13 16:02:34 vollmer Exp $
%% -*- coding: utf-8 -*-
%%%%%%%%%%%%%%%%%%%%%%%%%%%%%%%%%%%%%%%%%%%%%%%%%%%%%%%%%%%%%%%%%%%%%%%%%%%%%%%
% Abstand zwischen Paragraphen und Einzug entfernen
\setlength{\parskip}{1em}  % Abstand zwischen Paragraphen (1em = Schriftgröße)
\setlength{\parindent}{0pt}  % Kein Einzug am Anfang eines Paragraphen

\chapter{Projektablaufplan}
\textbf{Zeitraum:} 01.10.2025 – 04.05.2026

Der Projektzeitraum umfasst zwei Semester und deckt alle Phasen von der Vorbereitung über die Konzeptentwicklung und Prototyping bis hin zu Testing, Optimierung und Abschlussdokumentation ab.

\begin{figure}[H]
\centering
\fbox{\includegraphics[width=\textwidth]{gantt_diagramm}}
\caption{Projektablaufplan (Gantt-Diagramm) des KI-gestützten Schreib-Tutors}
\label{fig:gantt}
\end{figure}
\textbf{Legende:}
Gelb markierte Einträge kennzeichnen die definierten Meilensteine des Projekts.
Hellblau hinterlegte Aufgaben stellen Maßnahmen zur Qualitätssicherung dar.
Die in Hellgrau dargestellten Balken sowie die vertikalen Trennlinien visualisieren die
übergeordneten Projektphasen. Alle übrigen Balken entsprechen regulären Arbeitspaketen.




\paragraph{Projektvorbereitung (01.10.2025 -- 25.10.2025)}
Infrastruktur und Organisation für das Projekt aufbauen, Team bilden und Aufgaben verteilen, Entwicklungsumgebung einrichten.
\begin{itemize}
    \item \textbf{Themen- und Gruppenfindung (01.10.2025 -- 10.10.2025):} Bildung des Projektteams, Auswahl des Projektthemas innerhalb des Teams, Zuweisung von Rollen innerhalb der Gruppe.
    \item \textbf{QS-Maßnahmen: Besprechung mit Betreuer (06.10.2025 -- 10.10.2025):} Frühzeitige Rücksprache mit Betreuer zu Zielen, Anforderungen und Planung.
    \item \textbf{Meilenstein: Themenmitteilung (10.10.2025):} Offizielle Einreichung der Themenmitteilung des Projektes an der DHBW.
    \item \textbf{Einrichtung der Arbeitsumgebung (10.10.2025 -- 25.10.2025):} Installation und Konfiguration der Entwicklungsumgebungen, Erstellung eines GitHub-Repositories sowie Einrichtung der erforderlichen Schreibwerkzeuge.
    \item \textbf{Meilenstein: Kick-off-Besprechung (25.10.2025):} Besprechung mit Projektteam und Betreuer zum offiziellen Start des Projektes.
\end{itemize}

\paragraph{Konzeptentwicklung (26.10.2025 -- 07.12.2025)}
Funktionen und Systemarchitektur des KI-Schreib-Tutors konzipieren, didaktische Leitlinien und Interaktionslogik festlegen.
\begin{itemize}
    \item \textbf{Spezifikation der Funktionen (26.10.2025 -- 07.11.2025):} Lasten- und Pflichtenheft erstellen, Systemziele und Anforderungen detailliert ausarbeiten.
    \item \textbf{Recherche zu KI-Modellen (08.11.2025 -- 21.11.2025):} Geeignete Sprachmodelle, Feedbackmethoden und Dialogstrategien auswählen.
    \item \textbf{QS-Maßnahmen: Risikoanalyse (14.11.2025 -- 21.11.2025):} Untersuchung potenziell auftretender Risiken sowie Bewertung von deren Auswirkungen und Eintrittswahrscheinlichkeit.
    \item \textbf{Vergleich verschiedener Ansätze (22.11.2025 -- 07.12.2025):} Analyse und Bewertung verschiedener KI-Modelle, Methoden zur Rückfragegenerierung und Feedbackstrategien.
    \item \textbf{Meilenstein: Erstes Kozept (07.12.2025):} Erstes System- und Funktionskonzept abgestimmt.
\end{itemize}

\paragraph{Prototyping (08.12.2025 -- 13.02.2026)}
Entwicklung und Integration der ersten Softwaremodule für den KI-Schreib-Tutor, Implementierung der dialogischen Rückfragen und grundlegender Feedbackmechanismen.
\begin{itemize}
    \item \textbf{Implementierung Software-Prototyp (08.12.2025 -- 23.12.2025):} Entwicklung der Basismodule, Integration der Feedbacklogik, Textanalyse und Datenverarbeitung.
    \item \textbf{GUI-Integration und Dialoglogik (24.12.2025 -- 20.01.2026):} Erstellung der Benutzeroberfläche, Implementierung der Dialogstrukturen und Eingabefunktionen.
    \item \textbf{Integration bestehender Komponenten (21.01.2026 -- 13.02.2026):} Zusammenführung aller Module, Anpassung bestehender Codes, Optimierung der Systemstabilität.
    \item \textbf{QS-Maßnahmen: Testing (10.01.2026 -- 13.02.2026):} Funktionstests, Code-Reviews und iterative Verbesserungen zur Sicherstellung der Softwarequalität.
    \item \textbf{Meilenstein: Erster Prototyp (13.02.2026):} Fertigstellung des ersten lauffähigen Prototyps des KI-Tutors.
\end{itemize}

\paragraph{Testing, Evaluation und Optimierung (14.02.2026 -- 31.03.2026)}
Umfassende Testphase zur Evaluation der Funktionalität, Usability und Lernwirksamkeit des KI-Schreib-Tutors sowie Umsetzung notwendiger Verbesserungen.
\begin{itemize}
    \item \textbf{System- und Feldtests (14.02.2026 -- 28.02.2026):} Überprüfung der Software im realen Nutzungskontext, Sammlung von Feedback durch Studierende.
    \item \textbf{Analyse der Testergebnisse (01.03.2026 -- 10.03.2026):} Auswertung der Testergebnisse, Identifikation von Schwachstellen und Verbesserungspotential.
    \item \textbf{Verbesserungsschleifen (11.03.2026 -- 31.03.2026):} Optimierung der KI-Dialoge, Verbesserung der Argumentations-Feedbacklogik und Anpassung der Benutzeroberfläche.
    \item \textbf{QS-Maßnahmen: Evaluation (14.02.2026 -- 31.03.2026):} Kontinuierliche Überprüfung der Lernförderlichkeit, Begleitung der Test- und Optimierungsprozesse, Dokumentation aller Anpassungen und Qualitätssicherung.
    \item \textbf{Meilenstein: Ende Optimierung (31.03.2026):} Abschluss der Test- und Optimierungsphase.
\end{itemize}

\paragraph{Abschluss und Dokumentation (01.04.2026 -- 04.05.2026)}
Erstellung der abschließenden Projektdokumentation, Reflexion der Ergebnisse und Vorbereitung der Präsentation des KI-Schreib-Tutors.
\begin{itemize}
    \item \textbf{Erstellung Abschlussbericht (01.04.2026 -- 20.04.2026):} Zusammenfassung der Projektergebnisse, wissenschaftliche Auswertung und Reflexion der Zielerreichung.
    \item \textbf{Vorbereitung Präsentation (21.04.2026 -- 04.05.2026):} Erstellung der Präsentationsunterlagen, Demonstration des Prototyps und Feinschliff der Darstellung.
    \item \textbf{Meilenstein: Abschluss (04.05.2026):} Projektabschluss mit finaler Abgabe und Präsentation.
\end{itemize}

%%%%%%%%%%%%%%%%%%%%%%%%%%%%%%%%%%%%%%%%%%%%%%%%%%%%%%%%%%%%%%%%%%%%%%%%%%%%%%%
