%%%%%%%%%%%%%%%%%%%%%%%%%%%%%%%%%%%%%%%%%%%%%%%%%%%%%%%%%%%%%%%%%%%%%%%%%%%%%%
%% Descr:       Vorlage für Berichte der DHBW-Karlsruhe, Ein Kapitel
%% Author:      Prof. Dr. Jürgen Vollmer, vollmer@dhbw-karlsruhe.de
%% $Id: kapitel1.tex,v 1.24 2020/03/13 16:02:34 vollmer Exp $
%% -*- coding: utf-8 -*-
%%%%%%%%%%%%%%%%%%%%%%%%%%%%%%%%%%%%%%%%%%%%%%%%%%%%%%%%%%%%%%%%%%%%%%%%%%%%%%%
% Abstand zwischen Paragraphen und Einzug entfernen
\setlength{\parskip}{1em}  % Abstand zwischen Paragraphen (1em = Schriftgröße)
\setlength{\parindent}{0pt}  % Kein Einzug am Anfang eines Paragraphen

\chapter{Projektziel und Projektauftrag}
\section{Projektziel}
Das Ziel dieses Projektes ist die Entwicklung und Evaluation eines KI-gestützten Schreib-Tutors, der Studierende technischer Fachrichtungen beim Erwerb wissenschaftlicher Schreibkompetenzen unterstützt.

Im Vordergrund steht die Gestaltung eines interaktiven Systems, das nicht automatisch Texte generiert, sondern Lernprozesse fördert, indem es gezielte Rückfragen stellt, Argumentationsstrukturen stärkt und sprachlich-stilistische Sensibilität vermittelt.

Ziel ist die Erstellung eines funktionsfähigen Prototyps auf Basis aktueller Sprachmodelle sowie die Untersuchung didaktischer und ethischer Rahmenbedingungen für den Einsatz solcher Systeme im Hochschulkontext.
\section{Projektauftrag}
\begin{itemize}
\item \textbf{Name des Projektes:} & Entwicklung und Evaluation eines KI-gestützten Schreib-Tutors zur Förderung wissenschaftlicher Schreibkompetenz 
\item \textbf{Kurzbeschreibung:} & Entwicklung eines interaktiven, KI-basierten Schreib-Tutors, der Studierende beim wissenschaftlichen Schreiben begleitet, indem er gezieltes Feedback und Reflexionsimpulse gibt. 
\item \textbf{Identifikationsbegriff:} & KIWRT025
\item \textbf{Projektleiter:} & Stefan Schäfer
\item \textbf{Projektteam:} & Stefan Schäfer, Philip Sagawe
\item \textbf{Unterauftragnehmer:} & Kay Margarethe Berkling
\item \textbf{Zeitraum:} & Projekt nach Semesterplan
\item \textbf{Budget:} & 500,00 €
\item \textbf{Einsatzmittelkosten:} &
\begin{itemize}
    \item Nutzung von n8n – einmaliger Zugang über Studentenlizenz
    \item KI-Tokens – Pauschalbudget 300,00 €

Dies deckt den voraussichtlichen Verbrauch während der Entwicklungs- und Testphase ab.

Durchschnittliche Kosten: Input-Tokens ca. 0,25 € pro 1 Mio Tokens, Output-Tokens ca. 1,25 € pro 1 Mio Tokens.
\end{itemize}
\item \textbf{Geplante Meilensteine:} &
\begin{itemize}
    \item Erarbeitung des Systemkonzepts
    \item Integration von bestehenden KI-Komponenten und Basiscodes
    \item Entwicklung des ersten Prototyps
    \item KI-Feinabstimmung (Training und Optimierung)
    \item Modellvergleiche und Evaluation mit Studierenden
\end{itemize}
\item \textbf{Fertigstellungstermin:} & Ende des 6. Semesters, voraussichtlich 17. Mai 2026 \\
\end{itemize}
%%%%%%%%%%%%%%%%%%%%%%%%%%%%%%%%%%%%%%%%%%%%%%%%%%%%%%%%%%%%%%%%%%%%%%%%%%%%%%%
