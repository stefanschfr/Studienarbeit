%%%%%%%%%%%%%%%%%%%%%%%%%%%%%%%%%%%%%%%%%%%%%%%%%%%%%%%%%%%%%%%%%%%%%%%%%%%%%%
%% Descr:       Vorlage für Berichte der DHBW-Karlsruhe, Ein Kapitel
%% Author:      Prof. Dr. Jürgen Vollmer, vollmer@dhbw-karlsruhe.de
%% $Id: kapitel1.tex,v 1.24 2020/03/13 16:02:34 vollmer Exp $
%% -*- coding: utf-8 -*-
%%%%%%%%%%%%%%%%%%%%%%%%%%%%%%%%%%%%%%%%%%%%%%%%%%%%%%%%%%%%%%%%%%%%%%%%%%%%%%%
% Abstand zwischen Paragraphen und Einzug entfernen
\setlength{\parskip}{1em}  % Abstand zwischen Paragraphen (1em = Schriftgröße)
\setlength{\parindent}{0pt}  % Kein Einzug am Anfang eines Paragraphen

\chapter{Produkt-/Systemdefinition}
\section{Lastenheft}
\begin{itemize}
    \item Das System soll Studierende beim wissenschaftlichen Schreiben unterstützen und Lernprozesse fördern.
    \item Der Schreib-Tutor soll KI-gestützte Rückfragen stellen und Feedback zur Argumentationsstruktur geben.
    \item Unterstützung der Studierenden in sprachlich-stilistischer Sensibilität und wissenschaftlicher Ausdrucksweise.
    \item Bereitstellung eines prototypischen Tutors, der dialogisch arbeitet, ohne Texte automatisch zu generieren.
    \item Hohe Nutzerfreundlichkeit bei intuitiver Bedienung.
    \item Einhaltung aller datenschutzrechtlichen und ethischen Anforderungen.
    \item Ziel ist ein funktionierender Demonstrator für Tests mit Studierenden und Präsentation.
\item Projektdauer: 2 Semester bis 17.05.2026, Budget ca. 500,00 €.
\end{itemize}
\section{Pflichtenheft}
\begin{itemize}
    \item Implementierung eines KI-Moduls auf Basis aktueller Sprachmodelle zur Generierung von Rückfragen und Feedback.
    \item Entwicklung einer Software-Plattform zur Integration des KI-Moduls und Interaktion mit Studierenden.
    \item Erstellung einer grafischen Benutzeroberfläche, die Eingaben von Studierenden aufnimmt und KI-Feedback anzeigt.
    \item Implementierung von Dialogmechanismen zur Förderung von Argumentationslogik und wissenschaftlichem Stil.
    \item Evaluierung der Wirksamkeit durch Tests mit Studierenden (kurze Schreibübungen, Feedbackauswertung).
    \item Dokumentation und Transparenz der KI-Nutzung im Sinne von DFG- und Hochschulrichtlinien.
    \item Architektur offenhalten für künftige Funktionserweiterungen, z.B. weitere Interaktionsformen oder zusätzliche Schreibübungsmodi.
\end{itemize}
%%%%%%%%%%%%%%%%%%%%%%%%%%%%%%%%%%%%%%%%%%%%%%%%%%%%%%%%%%%%%%%%%%%%%%%%%%%%%%%