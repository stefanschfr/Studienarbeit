%%%%%%%%%%%%%%%%%%%%%%%%%%%%%%%%%%%%%%%%%%%%%%%%%%%%%%%%%%%%%%%%%%%%%%%%%%%%%%
%% Descr:       Vorlage für Berichte der DHBW-Karlsruhe, Ein Kapitel
%% Author:      Prof. Dr. Jürgen Vollmer, vollmer@dhbw-karlsruhe.de
%% $Id: kapitel1.tex,v 1.24 2020/03/13 16:02:34 vollmer Exp $
%% -*- coding: utf-8 -*-
%%%%%%%%%%%%%%%%%%%%%%%%%%%%%%%%%%%%%%%%%%%%%%%%%%%%%%%%%%%%%%%%%%%%%%%%%%%%%%%
% Abstand zwischen Paragraphen und Einzug entfernen
\setlength{\parskip}{1em}  % Abstand zwischen Paragraphen (1em = Schriftgröße)
\setlength{\parindent}{0pt}  % Kein Einzug am Anfang eines Paragraphen

\chapter{Protokoll 2}
\textbf{Datum:} & 03.11.2025 \\
\textbf{Dauer:} & 2 Stunden \\
\textbf{Ort:} & IKEA Restaurant \\
\textbf{Teilnehmer:} & Stefan Schäfer, Philip Sagawe \\

\section{Besprochene Inhalte}

\subsection{Definition des Minimal Viable Product (MVP)}
\begin{itemize}
    \item Austausch über inhaltliche und technische Ausrichtung des MVP
    \item Diskussion möglicher Funktionsbereiche und deren Relevanz
    \item Bewertung der Umsetzbarkeit und Priorisierung potenzieller Funktionen
    \item Ziel: gemeinsames Verständnis der grundlegenden Anforderungen für die erste Entwicklungsphase
\end{itemize}

\subsection{Abstimmung zur weiteren Vorgehensweise}
\begin{itemize}
    \item Vereinbarung, die diskutierten Funktionsideen in einem separaten Dokument zusammenzufassen
    \item Geplante Vorlage des Dokuments zur Rückmeldung an die Betreuerin
    \item Festlegung, den endgültigen Funktionsumfang auf Basis der Rückmeldung zu definieren
\end{itemize}

\subsection{Kommunikation mit der Betreuerin}
\begin{itemize}
    \item Vorbereitung einer E-Mail an Prof. Dr. Berkling zur Klärung ihrer Vorstellungen und Erwartungen
    \item Anfrage nach einem möglichen zusätzlichen Abstimmungstermin
    \item Ziel: detaillierte Besprechung der konzeptionellen und funktionalen Schwerpunkte
\end{itemize}


\section{Aufgaben und nächste Schritte}
\begin{table}[H]
    \centering
    \caption{Abgeleitete Aufgaben aus dem Treffen zur MVP-Definition}
    \begin{tabular}{|p{4cm}|p{4cm}|p{3cm}|p{3cm}|}
        \hline
        \textbf{Aufgabe} & \textbf{Beschreibung} & \textbf{Verantwortlich} & \textbf{Status} \\ \hline

        Zusammenfassung der Diskussionsinhalte &
        Erstellung einer kurzen Dokumentation der im Treffen diskutierten Themen und Funktionsideen &
        Projektteam &
        In Bearbeitung \\ \hline

        Rückmeldung von Berkling einholen &
        Versand der E-Mail zur Klärung der Erwartungen und Einplanung eines Folgetermins &
        Stefan Schäfer &
        In Bearbeitung \\ \hline

        Technische Machbarkeitsanalyse &
        Erste Einschätzung zur Realisierbarkeit der diskutierten Funktionen &
        Philip Sagawe &
        Offen \\ \hline

        Vorbereitung Folgetermin &
        Aufbereitung der bisherigen Ergebnisse und Abstimmungspunkte für das Gespräch mit der Betreuerin &
        Projektteam &
        Ausstehend \\ \hline

    \end{tabular}
    \label{tab:aufgaben_mvp_diskussion}
\end{table}

\section{Nächster Termin}
Ein weiterer Termin mit Prof. Dr. Berkling wird nach Erhalt ihrer Rückmeldung per E-Mail abgestimmt.
Der genaue Zeitpunkt ist derzeit noch offen und richtet sich nach dem Fortschritt bei der Ausarbeitung der MVP-Definition.


%%%%%%%%%%%%%%%%%%%%%%%%%%%%%%%%%%%%%%%%%%%%%%%%%%%%%%%%%%%%%%%%%%%%%%%%%%%%%%%
