%%%%%%%%%%%%%%%%%%%%%%%%%%%%%%%%%%%%%%%%%%%%%%%%%%%%%%%%%%%%%%%%%%%%%%%%%%%%%%
%% Descr:       Vorlage für Berichte der DHBW-Karlsruhe, Ein Kapitel
%% Author:      Prof. Dr. Jürgen Vollmer, vollmer@dhbw-karlsruhe.de
%% $Id: kapitel1.tex,v 1.24 2020/03/13 16:02:34 vollmer Exp $
%% -*- coding: utf-8 -*-
%%%%%%%%%%%%%%%%%%%%%%%%%%%%%%%%%%%%%%%%%%%%%%%%%%%%%%%%%%%%%%%%%%%%%%%%%%%%%%%
% Abstand zwischen Paragraphen und Einzug entfernen
\setlength{\parskip}{1em}  % Abstand zwischen Paragraphen (1em = Schriftgröße)
\setlength{\parindent}{0pt}  % Kein Einzug am Anfang eines Paragraphen

\chapter{Protokoll 1}
\textbf{Datum:} & 07.10.2025 \\
\textbf{Dauer:} & 45 Minuten \\
\textbf{Ort:} & DHBW Karlsruhe \\
\textbf{Teilnehmer:} & Stefan Schäfer, Philip Sagawe, Kay Margarethe Berkling \\

\section{Besprochene Inhalte}

\subsection{Überprüfung der Themenmitteilung}
\begin{itemize}
    \item Prüfung der Themenmitteilung auf Übereinstimmung mit Anforderungen
    \item Klärung, ob Anpassungen nötig sind
    \item Festlegung der hervorgehobenen Punkte für Genehmigung
\end{itemize}

\subsection{Projektumfang}
\begin{itemize}
    \item Diskussion des geplanten Umfangs der Projektarbeit
    \item Hinweise zu Inhalten und Arbeitspaketen
    \item Klärung realistischer zeitlicher Ressourcen
\end{itemize}

\subsection{Umsetzungsvorstellungen}
\begin{itemize}
    \item Anforderungen an die Funktionalitäten des KI-Schreib-Tutors
    \item Hinweise zur Prototypentwicklung
    \item Erwartungen hinsichtlich Evaluation und Qualität
    \item Besondere Anforderungen an Systemarchitektur
\end{itemize}

\section{Aufgaben und nächste Schritte}
\begin{table}[H]
    \centering
    \caption{Abgeleitete Aufgaben aus dem Termin zur Themenmitteilung}
    \begin{tabular}{|p{4cm}|p{4cm}|p{3cm}|p{3cm}|}
        \hline
        \textbf{Aufgabe} & \textbf{Beschreibung} & \textbf{Verantwortlich} & \textbf{Status} \\ \hline

        Themenmitteilung überarbeiten &
        Einarbeitung der Anmerkungen aus dem Gespräch zur formalen und inhaltlichen Präzisierung &
        Projektteam &
        Abgeschlossen \\ \hline

        Projektziele konkretisieren &
        Detaillierte Definition der angestrebten Funktionen und des didaktischen Nutzens des KI-Schreib-Tutors &
        Projektteam &
        In Bearbeitung \\ \hline

        Arbeitspakete strukturieren &
        Aufteilung des Projekts in logische Module inkl. zeitlicher Planung &
        Projektteam &
        Offen \\ \hline

        Technische Umsetzung abstimmen &
        Vorläufige Festlegung der Systemarchitektur und genutzten KI-Technologien &
        Projektteam &
        Offen \\ \hline

        Nächste Abstimmung planen &
        Termin bei Bedarf per E-Mail mit Betreuerin koordinieren &
        Projektteam &
        Ausstehend \\ \hline

    \end{tabular}
    \label{tab:aufgaben_besprechung}
\end{table}

\section{Nächster Termin}
Ein weiterer Abstimmungstermin mit der Betreuerin wird bei Bedarf per E-Mail koordiniert.
Der genaue Zeitpunkt steht derzeit noch aus und wird in Abhängigkeit vom Fortschritt
der Überarbeitung der Themenmitteilung festgelegt.


%%%%%%%%%%%%%%%%%%%%%%%%%%%%%%%%%%%%%%%%%%%%%%%%%%%%%%%%%%%%%%%%%%%%%%%%%%%%%%%
