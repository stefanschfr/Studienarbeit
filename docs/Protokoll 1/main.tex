%&bericht

%%%%%%%%%%%%%%%%%%%%%%%%%%%%%%%%%%%%%%%%%%%%%%%%%%%%%%%%%%%%%%%%%%%%%%%%%%%%%%%
%% Descr:       Vorlage für Berichte der DHBW-Karlsruhe
%% Author:      Prof. Dr. Jürgen Vollmer, juergen.vollmer@dhbw-karlsruhe.de
%% $Id: bericht.tex,v 1.25 2020/03/13 15:07:45 vollmer Exp $
%%  -*- coding: utf-8 -*-
%%%%%%%%%%%%%%%%%%%%%%%%%%%%%%%%%%%%%%%%%%%%%%%%%%%%%%%%%%%%%%%%%%%%%%%%%%%%%%%

\documentclass[
   ngerman          % neue deutsche Rechtschreibung
  ,a4paper          % Papiergrösse
% ,twoside          % Zweiseitiger Druck (rechts/links)
% ,10pt             % Schriftgrösse
% ,11pt
  ,12pt
  ,pdftex
%  ,disable         % Todo-Markierungen auschalten
]{report}

% Bitte die Codierung Ihrer Dateien auswählen:
% \usepackage[latin1]{inputenc}    % Für UNIX mit ISO-LATIN-codierten Dateien
% \usepackage[applemac]{inputenc}  % Für Apple Mac
% \usepackage[ansinew]{inputenc}   % Für Microsoft Windows
\usepackage[utf8]{inputenc}        % UTF-8 codierte Dateien
                                   % Dieses Dokument ist unter Unix erstellt, daher
                                   % wird diese Input-Codierung benutzt.
\usepackage[onehalfspacing]{setspace}
\usepackage{bericht}
\usepackage{pdflscape}

%% ACHTUNG, wenn man eine eigene Formatdatei (bericht.fmt) benutzt, werden Änderungen an bericht.sty
%% erst wirksam, wenn die Format-Datei neu erzeugt wurde!!!
%% Genauer alle Änderungen, die textuell vor der nächsten Zeile ".... endofdump...." stehen
%% werden erst wirksam, wenn die Formatdatei neu erzeugt wurde
\csname endofdump\endcsname

%%%%%%%%%%%%%%%%%%%%%%%%%%%%%%%%%%%%%%%%%%%%%%%%%%%%%%%%%%%%%%%%%%%%%%%%%%%%%%%
%% Angaben zur Arbeit
%%%%%%%%%%%%%%%%%%%%%%%%%%%%%%%%%%%%%%%%%%%%%%%%%%%%%%%%%%%%%%%%%%%%%%%%%%%%%%%

\newcommand{\Autor}{Stefan Schäfer}
\newcommand{\Projektleiter}{Stefan Schäfer}
\newcommand{\Kursbezeichnung}{TINF23B5}

\newcommand{\FirmenName}{Weisenburger Bau GmbH}
\newcommand{\FirmenStadt}{Karlsruhe}
\newcommand{\FirmenLogoDeckblatt}{}

% Falls es kein Firmenlogo gibt:
%  \newcommand{\FirmenLogoDeckblatt}{}

\newcommand{\BetreuerFirma}{Titel Vorname Nachname}
\newcommand{\Team}{Stefan Schäfer, Philip Sagawe}

%%%%%%%%%%%%%%%%%%%%%%%%%%%%%%%%%%%%%%%%%%%%%%%%%%%%%%%%%%%%%%%%%%%%%%%%%%%%%%%%%%%%%

% Wird auf dem Deckblatt und in der Erklärung benutzt:
%\newcommand{\Was}{Projekt-/Studien-/Bachleorarbeit}
\newcommand{\Was}{Studienarbeit}
%\newcommand{\Was}{Studienarbeit}
%\newcommand{\Was}{Bachleorarbeit}

%%%%%%%%%%%%%%%%%%%%%%%%%%%%%%%%%%%%%%%%%%%%%%%%%%%%%%%%%%%%%%%%%%%%%%%%%%%%%%%%%%%%%

\newcommand{\Titel}{Protokoll 1}
\newcommand{\AbgabeDatum}{}

\newcommand{\Dauer}{}

% \newcommand{\Abschluss}{Bachelor of Engineering}
\newcommand{\Abschluss}{Technisch Wissenschaftliches Arbeiten}

\newcommand{\Studiengang}{Informatik}
% \newcommand{\Studiengang}{Informatik / Angewandte Informatik}

\hypersetup{%%
  pdfauthor={\Autor},
  pdftitle={\Titel},
  pdfsubject={\Was}
}

%%%%%%%%%%%%%%%%%%%%%%%%%%%%%%%%%%%%%%%%%%%%%%%%%%%%%%%%%%%%%%%%%%%%%%%%%%%%%%%

% Wenn \includeonly{..} benutzt wird, werden nur diese Kaptitel ausgegeben.
\includeonly{
  abk
 ,kapitel1
}

%%%%%%%%%%%%%%%%%%%%%%%%%%%%%%%%%%%%%%%%%%%%%%%%%%%%%%%%%%%%%%%%%%%%%%%%%%%%%%%

% Benutzt man das "biblatex"-Paket, dann muß das hier stehen:
% siehe auch die mit BIBLATEX markierten Zeilen in bericht.sty
\bibliography{bericht}

\begin{document}

%%%%%%%%%%%%%%%%%%%%%%%%%%%%%%%%%%%%%%%%%%%%%%%%%%%%%%%%%%%%%%%%%%%%%%%%%%%%%%%

\begin{titlepage}
\begin{center}
\vspace*{-2cm}
\FirmenLogoDeckblatt\hfill\includegraphics[width=4cm]{dhbw-logo}\\[2cm]
{\Huge \Titel}\\[1cm]
{\large \scshape \Was}\\[0.5cm]
{\large des Studienganges \Studiengang}\\[0.5cm]
{\large an der}\\[0.5cm]
{\large Dualen Hochschule Baden-Württemberg Karlsruhe}\\[0.5cm]
{\large von}\\[0.5cm]
{\large\bfseries \Autor}\\[1cm]
\vfill
\end{center}
\begin{tabular}{l@{\hspace{2cm}}l}
Kurs			         & \Kursbezeichnung		\\
Projektleiter	                 & \Projektleiter		\\
Team	 & \Team		\\
\end{tabular}
\end{titlepage}


%%%%%%%%%%%%%%%%%%%%%%%%%%%%%%%%%%%%%%%%%%%%%%%%%%%%%%%%%%%%%%%%%%%%%%%%%%%%%%%


\newpage
\tableofcontents           % Inhaltsverzeichnis hier ausgeben


% Jetzt kommt der "eigentliche" Text
%%%%%%%%%%%%%%%%%%%%%%%%%%%%%%%%%%%%%%%%%%%%%%%%%%%%%%%%%%%%%%%%%%%%%%%%%%%%%%
%% Descr:       Vorlage für Berichte der DHBW-Karlsruhe, Ein Kapitel
%% Author:      Prof. Dr. Jürgen Vollmer, vollmer@dhbw-karlsruhe.de
%% $Id: kapitel1.tex,v 1.24 2020/03/13 16:02:34 vollmer Exp $
%% -*- coding: utf-8 -*-
%%%%%%%%%%%%%%%%%%%%%%%%%%%%%%%%%%%%%%%%%%%%%%%%%%%%%%%%%%%%%%%%%%%%%%%%%%%%%%%
% Abstand zwischen Paragraphen und Einzug entfernen
\setlength{\parskip}{1em}  % Abstand zwischen Paragraphen (1em = Schriftgröße)
\setlength{\parindent}{0pt}  % Kein Einzug am Anfang eines Paragraphen

\chapter{Protokoll 3}
\textbf{Datum:} & 09.11.2025 \\
\textbf{Dauer:} & 45 Minuten \\
\textbf{Ort:} & Online \\
\textbf{Teilnehmer:} & Stefan Schäfer, Philip Sagawe \\

\section{Besprochene Inhalte}

\subsection{Marktanalyse}
\begin{itemize}
    \item Durchführung einer gemeinsamen Marktanalyse relevanter KI-Modelle und -Lösungen
    \item Vergleich bestehender Systeme hinsichtlich Leistungsfähigkeit, Kostenstruktur und Anwendungsbereich
    \item Analyse aktueller Markttrends im Bereich KI-gestützter Textanalyse und Schreibassistenz
    \item Ziel: Identifikation von Technologien, die als Referenz oder Grundlage für die eigene Entwicklung dienen können
\end{itemize}


\section{Aufgaben und nächste Schritte}
\begin{table}[H]
    \centering
    \caption{Abgeleitete Aufgaben aus dem Treffen zur Marktanalyse und Modellbewertung}
    \begin{tabular}{|p{4cm}|p{4cm}|p{3cm}|p{3cm}|}
        \hline
        \textbf{Aufgabe} & \textbf{Beschreibung} & \textbf{Verantwortlich} & \textbf{Status} \\ \hline

        Dokumentation der Marktanalyse &
        Zusammenfassung der untersuchten KI-Modelle, Kostenstrukturen und Bewertungsergebnisse in einem separaten Dokument &
        Projektteam &
        In Bearbeitung \\ \hline

        Auswahlvorschlag für Prototyp-Modell erstellen &
        Erstellung einer Empfehlung für das Modell, das im MVP eingesetzt werden soll, basierend auf Leistung, Kosten und Datenschutz &
        Stefan Schäfer &
        Ausstehend \\ \hline

    \end{tabular}
    \label{tab:aufgaben_marktanalyse}
\end{table}

\section{Nächster Termin}
Der nächste Abstimmungstermin zwischen Philip Sagawe und Stefan Schäfer ist für den 17.11.2025 vorgesehen.
Der genaue Ort des Treffens wird noch festgelegt und zu einem späteren Zeitpunkt abgestimmt.



%%%%%%%%%%%%%%%%%%%%%%%%%%%%%%%%%%%%%%%%%%%%%%%%%%%%%%%%%%%%%%%%%%%%%%%%%%%%%%%



\printindex
\addcontentsline{toc}{chapter}{Literaturverzeichnis}

% Haben Sie das "biblatex"-Paket nicht installiert, benutzen Sie folgendes:
% Ohne das "biblatex"-Paket (s. bericht.sty) produziert folgendes
% "deutsche" Zitate in Literaturverzeichnissen gemaß der Norm DIN 1505,
% Teil 2 vom Jan. 1984.
% Die Zitatmarken werden alphabetisch nach Verfassern
% sortiert und sind durch abgekürzte Verfasserbuchstaben plus
% Erscheinungsjahr in eckigen Klammern gekennzeichnet.

% \bibliographystyle{alphadin}
% \bibliography{bericht}

%%%%%%%%%%%%%%%%%%%%%%%%%%%%%%%%%%%%%%%5
% BIBLATEX
% Benutzt man das "biblatex"-Paket, muß man folgendes schreiben:
\def\refname{Literaturverzeichnis}
\printbibliography
%%%%%%%%%%%%%%%%%%%%%%%%%%%%%%%%%%%%%%%5


\end{document}
